\chapter{Chromatin immunoprecepitation}
\section{Experiment design}
Chromatin immunoprecepitation followed by sequencing (ChIP-seq) is a method developed for the genome wide identification of transcription factor (TF) binding sites and epigenetic marks.
The protocol has many steps invilving sample preparation and analysis.
DNA-binding protein is crosslinked to DNA \emph{in vivo} with formaldehyde.
After that the chromatin fragmentation using sonication makes DNA fragments in the 200-600bp range.

Quality of the antibody used in the ChiP-seq assay influence the value of ChIP-seq data.
Antibody specifity will give a high level of enrichment compared to the background. 
It makes available to detect binding events.
A typical experiment requires 10-100 ng of DNA.

\section{Sequencing library quality plays the critical role for next-generation sequencing}
The fragmentation of the target DNA is a key step for NGS library construction.
Physical metods or enzymatic methods are typically used for DNA fragmentation~\cite{}.
%comparison of both phys and enzyme?
After fragmentation the sequencing oligonucleotyde adapters of a constant length with specific sequences are ligated.
Those sequences are specially designed to interact with the NGS platform such as Illumina or Ion Torrent.
Determination of the library size referring to the insert size which is limited by the high-throughput sequencing instrumentation.
Process of cluster generation, after the fragments have attached, constrains the optimal insert size in Illumina technology.
Shorter products amplify more efficiently.





\section{Sequencing}
Purified DNAs fragments from ChIP-seq samples are sequenced as reads of length 36-100bp and can be uniquely aligned to the genome reference.
Different next-generation platforms can be utilized.
The technology provides siquencing of both single-end and paired-end reads.
A single run of the Illumina/Solexa sequencing technology generetas 50-200 million reads\cite{park2009chip}.
Single-end reads are usrd for common ChIP-seq analysis, while paired-end are evolved to improve the library complexity.
That helps to focuse at repetitive regions \cite{chen2012systematic}.

\section{Mapping}
Sequenced short reads are represented as a text file and can be aligned with mapping software based on Burrows-Wheeler transform algorithm \cite{li2009fast} \cite{siren2014indexing}
