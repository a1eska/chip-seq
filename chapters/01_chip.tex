\chapter{Chromatin immunoprecepitation}
\section{Experiment design}
Chromatin immunoprecepitation followed by sequencing (ChIP-seq) is a method developed for the genome wide identification of transcription factor (TF) binding sites and epigenetic marks.
The protocol has many steps invilving sample preparation and analysis.
DNA-binding protein is crosslinked to DNA \emph{in vivo} with formaldehyde.
After that the chromatin fragmentation using sonication makes DNA fragments in the 200-600bp range.

Quality of the antibody used in the ChiP-seq assay influence the value of ChIP-seq data.
Antibody specifity will give a high level of enrichment compared to the background. 
It makes available to detect binding events.
A typical experiment requires 10-100 ng of DNA.


\section{Sequencing}
Purified DNAs fragments from ChIP-seq samples are sequenced as reads of length 36-100bp and can be uniquely aligned to the genome reference.
Different next-generation platforms can be utilized.
The technology provides siquencing of both single-end and paired-end reads.
A single run of the Illumina/Solexa sequencing technology generetas 50-200 million reads\cite{park2009chip}.
Single-end reads are usrd for common ChIP-seq analysis, while paired-end are evolved to improve the library complexity.
That helps to focuse at repetitive regions \cite{chen2012systematic}.

\section{Mapping}
Sequenced short reads are represented as a text file and can be aligned with mapping software based on Burrows-Wheeler transform algorithm \cite{li2009fast} \cite{siren2014indexing}
