\chapter{Chromatin immunoprecepitation}

\section{Intro}
Chromatin immunoprecipitation followed by sequencing, also known as ChIP-sequencing or ChIP-seq, is a modern and relatively cheap method to analyze any protein associated with DNA. 
The most frequently investigated chromatin-associated proteins are transcription factors (TF), modifies histones, components of the transcriptional machinery, and chromatin-modifying enzymes.
Advances in high-throughput parallel sequencing technology and computational methods enable generate and analyze extremely large data sets. 
Unlike ChIP-on-chip, ChIP-seq was developed to understand genome-wide profiling and does not rely on knowledge of exact binding loci.~\cite{park2009chip} .


\section{Experiment design}

A typical protocol has many steps. 
The consideration od the first step depends on the properties of the protein under investigation. 
For example, histone-DNA interactions are strong enough. 
Thus the fixation using formaldehyde as a crosslinking agent may not be necessary~\cite{barski2008identification}. 
On the other hand, proteins associated with DNA for a short time require a crosslinking step. 
In the case of histone deacetylases (HDACs) and histone acetyltransferases (HATs), an additional disuccinimidyl glutarate (DSG) treatment step before formaldehyde crosslinking demands~\cite{wang2009genome}. 

Cross-linked chromatin is fragmented before ChIP. 
Fragmentation way is depending on the purpose of the experiment, cell type, number of cells, fixation conditions. 
For nucleosome modifications, MNase digestion may be preferred~\cite{kidder2011chip}.  
The method allows generating high-resolution data of mononucleosome sized particles. 
However, the nucleosome instability may cause the loss of signal.
To identify TF binding events, sonication of crosslinked chromatin is a preferable method. 
In this case, the micrococcal nuclease may cause degradation of the linker DNA~\cite{kidder2011chip}.
The sonication conditions should be optimized for each experiment type. 
A Sonication buffer can influence the result~\cite{steger2008dot1l}. 
It is also important to avoid oversonication during the library preparation for transcription factors. 


Then DNA fragments of 150 to 300 bp long are selected and sequenced.
This size range is equivalent to mono- and dinucleosome chromatin fragments~cite{}.



Sequenced fragments referred to reads are then mapped to the reference. 
Enriched regions are compared with the control dataset.










\section{MOCK versus input}


\section{Sequencing library quality plays the critical role for next-generation sequencing}
The fragmentation of the target DNA is a key step for NGS library construction.
Physical metods or enzymatic methods are typically used for DNA fragmentation~\cite{}.
%comparison of both phys and enzyme?
After fragmentation the sequencing oligonucleotyde adapters of a constant length with specific sequences are ligated.
\todo{"constant length" - o velikost v podstate nejde, dulezita je sekvence adapteru}
Those sequences are specially designed to interact with the NGS platform such as Illumina or Ion Torrent.
Determination of the library size referring to the insert size which is limited by the high-throughput sequencing instrumentation.
Process of cluster generation, after the fragments have attached, constrains the optimal insert size in Illumina technology.
Shorter products amplify more efficiently.
\todo{"products" -> shorter fragments. Kratsi fragmenty take maji mensi sanci, ze dojde ke kontaktu se sousednim clusterem}

Library preparation from purified DNA for ChIP-seq experiment should be as complex as possible.
\todo{veta nedava dobry smysl - komplexni ma byt knihovna, nikoli priprava knihovny}
More starting material induces less amplification.
\todo{More starting material requires less amplification.}
Thus library complaxity is better.
\todo{complexity; v textu je hodne preklepu - muzes nejak vyuzit spellchecking?}


\section{Sequencing}
Purified DNAs fragments from ChIP-seq samples are sequenced as reads of length 36-100bp and can be uniquely aligned to the genome reference.
Different next-generation platforms can be utilized.
The technology provides siquencing of both single-end and paired-end reads.
A single run of the Illumina/Solexa sequencing technology generetas 50-200 million reads\cite{park2009chip}.
Single-end reads are usrd for common ChIP-seq analysis, while paired-end are evolved to improve the library complexity.
That helps to focuse at repetitive regions \cite{chen2012systematic}.

\section{Mapping}
Sequenced short reads are represented as a text file and can be aligned with mapping software based on Burrows-Wheeler transform algorithm \cite{li2009fast} \cite{siren2014indexing}
