\chapter*{Conclusion}
\addcontentsline{toc}{chapter}{Conclusion}

In the first chapter of the thesis we made a brief overview of ChIP-seq experimental design, and its computational analysis.
This method had been in use for a decade, however, it is still a powerful tool for understanding processes of the genome of interest.


The second chapter was about a computational tool called peak calling.
In this section we mentioned basic statistics utilized in peak calling evaluation. 
Also, we mentioned popular peak calling tools based on a different statistical model.
It is also good to know that even a well-designed tool has its limitations and may give rise to false-positive signals.

The last section was about known causes of false signals, which may be obtained during a ChIP-seq experiment.
Some false signals are caused by the nature of the genome. 
Others are raised due to sequencing and computational issues.
This section also contains a theoretical description of possible solutions. 
