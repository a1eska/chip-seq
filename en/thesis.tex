%%% The main file. It contains definitions of basic parameters and includes all other parts.

%% Settings for single-side (simplex) printing
% Margins: left 40mm, right 25mm, top and bottom 25mm
% (but beware, LaTeX adds 1in implicitly)
\documentclass[12pt,a4paper]{report}
\setlength\textwidth{145mm}
\setlength\textheight{247mm}
\setlength\oddsidemargin{15mm}
\setlength\evensidemargin{15mm}
\setlength\topmargin{0mm}
\setlength\headsep{0mm}
\setlength\headheight{0mm}
% \openright makes the following text appear on a right-hand page
\let\openright=\clearpage

%% Settings for two-sided (duplex) printing
% \documentclass[12pt,a4paper,twoside,openright]{report}
% \setlength\textwidth{145mm}
% \setlength\textheight{247mm}
% \setlength\oddsidemargin{14.2mm}
% \setlength\evensidemargin{0mm}
% \setlength\topmargin{0mm}
% \setlength\headsep{0mm}
% \setlength\headheight{0mm}
% \let\openright=\cleardoublepage

%% Generate PDF/A-2u
\usepackage[a-2u]{pdfx}

%% Character encoding: usually latin2, cp1250 or utf8:
\usepackage[utf8]{inputenc}

%% Prefer Latin Modern fonts
\usepackage{lmodern}

%% Further useful packages (included in most LaTeX distributions)
\usepackage{amsmath}        % extensions for typesetting of math
\usepackage{amsfonts}       % math fonts
\usepackage{amsthm}         % theorems, definitions, etc.
\usepackage{bbding}         % various symbols (squares, asterisks, scissors, ...)
\usepackage{bm}             % boldface symbols (\bm)
\usepackage{graphicx}       % embedding of pictures
\usepackage{fancyvrb}       % improved verbatim environment
\usepackage{natbib}         % citation style AUTHOR (YEAR), or AUTHOR [NUMBER]
\usepackage[nottoc]{tocbibind} % makes sure that bibliography and the lists
			    % of figures/tables are included in the table
			    % of contents
\usepackage{dcolumn}        % improved alignment of table columns
\usepackage{booktabs}       % improved horizontal lines in tables
\usepackage{paralist}       % improved enumerate and itemize
\usepackage{xcolor}         % typesetting in color
\usepackage{lineno}			% allow to number every line

%% My packages
\usepackage{todonotes}
%\usepackage{svg}
% \linenumbers
\linespread{1.25}
%%% Basic information on the thesis

% Thesis title in English (exactly as in the formal assignment)
\def\ThesisTitle{Sources, Identification and Removal of ChIP-seq Artifacts}

% Author of the thesis
\def\ThesisAuthor{Aleksandra Shumilova}

% Year when the thesis is submitted
\def\YearSubmitted{2021}

% Name of the department or institute, where the work was officially assigned
% (according to the Organizational Structure of MFF UK in English,
% or a full name of a department outside MFF)
\def\Department{Department of Cell Biology}

% Is it a department (katedra), or an institute (ústav)?
\def\DeptType{Department}

% Thesis supervisor: name, surname and titles
\def\Supervisor{RNDr. Martin Převorovský, Ph.D.}

% Supervisor's department (again according to Organizational structure of MFF)
\def\SupervisorsDepartment{Department of Cell Biology}

% Study programme and specialization
\def\StudyProgramme{Bioinformatics}
\def\StudyBranch{Bioinformatics}

% An optional dedication: you can thank whomever you wish (your supervisor,
% consultant, a person who lent the software, etc.)
\def\Dedication{%
I want to thank my supervisor Martin Převorovský. 
For his significant and valuable help and advice and for his time that he dedicated to our consultations. 
Out of the most valuable was the freedom in a decision he gave me during the whole process.
In certain situations, it is harder, yet more valuable not to be guided and to be forced to make my own decisions. 
It is my sincere belief that without his approach, the thesis would take a completely different path. 
}

% Abstract (recommended length around 80-200 words; this is not a copy of your thesis assignment!)
\def\Abstract{%


Chromatin immunoprecipitation is used to enrich DNA sequences that are associated with a protein of interest, and is used to map those sequences to the genomic regions.
Studying these DNA-protein binding regions provide an understanding of gene regulation and chromatin remodeling. 
However, some signals in fact represent no binding event and are known as false positives.
This thesis discusses the main sources of false-positive signals that commonly arise during ChIP-seq analysis, and offers possible solutions on how to minimize or filter them. 

Abstrakt:

Chromatinová immunoprecipitace se používá k obohacení sekvencí DNA, které jsou spojeny s požadovaným proteinem, a používá se k mapování těchto sekvencí na oblasti v genomu. Studium těchto vazebných míst proteinů na molekule DNA poskytuje pochopení genové regulace a remodelace chromatinu. Některé signály  nepředstavují žádné vazební místo a jsou známé jako falešně pozitiva. Tato práce se zabírá hlavními zdroji falešně pozitivních signálů, které běžně vznikají při analýze ChIP-seq dat, a nabízí možná řešení pro jejich snížení nebo filtrací.

}

% 3 to 5 keywords (recommended), each enclosed in curly braces
\def\Keywords{%
{ChIP-seq}, {chromatin imunoprecipitation}, {quality control}, {data filtration}
}

%% The hyperref package for clickable links in PDF and also for storing
%% metadata to PDF (including the table of contents).
%% Most settings are pre-set by the pdfx package.
\hypersetup{unicode}
\hypersetup{breaklinks=true}

% Definitions of macros (see description inside)
\include{macros}




% Title page and various mandatory informational pages
\begin{document}
\include{title}

%%% A page with an automatically generated table of contents of the bachelor thesis

\tableofcontents

%%% Abbreviations used in the thesis, if any, including their explanation
%%% In mathematical theses, it could be better to move the list of abbreviations to the beginning of the thesis.
\chapwithtoc{List of Abbreviations}

\begin{tabular}{c l}
	ChIP & chromatin immunoprecipitation \\
	ChIP-seq & ChIP followed by sequencing \\
	ChIP-on-chip & ChIP followed bygenomic tiling microarray hybridization \\
	DNA & deoxyribonucleic acid \\
	TF & transcription factor \\
	IP & immunoprecipitation \\
	HDAC & histone deacetylase \\
	HAT & histone acetyltransferase \\
	DSG & disuccinimidyl glutarate \\
	BWT & Burrows-Wheeler transform \\
	NGS & next-generation sequencing \\
	RNA & ribonucleic acid \\
	SNP & single nucleotide polymorphysm \\
	SSE & sequencing-specific error \\
	NRF & non-redundant fraction \\
	NSC & normalized cross-corelation \\
	RSC & relative cross-corelation \\
	QC & quality control \\
	IDR & irreproducible discovery rate \\
	FDR & false discovery rate \\
	dsDNA & double-stranded DNA \\
	ssDNA & single-stranded DNA \\
	HOT & high-occupancy target \\
	TSS & transcription start site \\
	KOIN & knockout implemented normalization \\
	PCR & polymerase chain reaction \\
	IgG & immunoglobulin G \\
	
\end{tabular}


%%% Each chapter is kept in a separate file
\chapter*{Introduction}
\addcontentsline{toc}{chapter}{Introduction}

Gene regulatory networks tend to be crucial in various genomic studies such as gene expression, cell segregation and differentiation, and disease~\cite{collas2010current}. 
Transcription factors (TFs) and chromatin modifiers are the main elements that control cellular functions by the dynamic association with target DNA within regulatory regions such as promoters and enhancers and their coding sequences.
Epigenetic modifications change the composition of the chromatin and modulate the association of the regulatory elements with DNA~\cite{antequera2003structure, kouzarides2007chromatin, mito2007histone}.
Several methods have been developed to identify biological regulatory elements and their behavior in the context of the interaction with chromatin and other transcription regulation components. 

Chromatin immunoprecipitation, known as ChIP, is a technique to investigate and analyze any protein associated with DNA inside the cell~\cite{o1995histone, o1996immunoprecipitation, nelson2006protocol}.
The most frequently investigated chromatin-associated proteins are transcription factors, post-translationally modified histones and histone variants in the genome, components of the transcriptional machinery, and chromatin-modifying enzymes.
Classical ChIP assays require a large number of cells because of the loss of material and extensive sample handling which leads to errors and inconsistency between replicates~\cite{o1996immunoprecipitation}. 
However, the introduction of the newer ChIP protocols promotes simplification of the procedure. 
Those simplifications enable the technique to be applied for a small amount of the cells and discover the genome-wide profiling of the target elements without prior knowledge of exact binding loci~\cite{collas2010current}. 

DNA microarrays in combination with ChIP assay was the first approach to identify and analyze \textit{in-vivo} the protein-chromatin interactions of interest in broad genome coverage. 
Such an approach is known as chromatin immunoprecipitation followed by genomic tiling microarray hybridization or simply ChIP-on-chip~\cite{ren2000genome, loden2005whole}. 
This technology produces highly reproducible profiles.
However, the large number of false positives and several limitations, such as genomic coverage, which is dependent on the microarray probe design, stimulated the development of other strategies. 

Chromatin immunoprecipitation followed by sequencing, known as ChIP-sequencing or ChIP-seq, is a modern and relatively cheap method to analyze any protein associated with DNA. 
Advances in high-throughput parallel sequencing technology and computational methods enable to generate and analyze extremely large data sets. 
Unlike Chip-on-chip, ChIP-seq generally builds a profile with better resolution and signal-to-noise ratio, and detect more signals~\cite{park2009chip}.
Also, the advantage of the ChIP-seq protocol is the small amount of the initial material~\cite{adli2010genome}.
\chapter{ChIP-seq as a High-Throughoutput Sequencing Technique.}
%\todo{ChIP-seq je spíš metoda než technologie}

%%%%%%%%%%%%%%%%%%%%%%%%%%%%%%%%%%%%%%%%%%%%%%%%%%%%%%5
% \section{Genome-Wide Methods for Protein Binding Sites on DNA.}
% Gene regulatory networks tend to be crucial in various genomic studies such as gene expression, cell segregation and differentiation, and disease~\cite{collas2010current}. 
% Transcription factors (TFs) and chromatin modifiers are the main elements that control cellular functions by the dynamic association with target DNA within regulatory regions such as promoters and enhancers and their coding sequences.
% Epigenetic modifications change the composition of the chromatin and modulate the association of the regulatory elements with DNA~\cite{antequera2003structure, kouzarides2007chromatin, mito2007histone}.
% Several methods have been developed to identify biological regulatory elements and their behavior in the context of the interaction with chromatin and other transcription regulation components. 

% Chromatin immunoprecipitation, known as ChIP, is a technique to investigate and analyze any protein associated with DNA inside the cell~\cite{o1995histone, o1996immunoprecipitation, nelson2006protocol}.
% The most frequently investigated chromatin-associated proteins are transcription factors, post-translationally modified histones and histone variants in the genome, components of the transcriptional machinery, and chromatin-modifying enzymes.
% Classical ChIP assays require a large number of cells because of the loss of material and extensive sample handling which leads to errors and inconsistency between replicates~\cite{o1996immunoprecipitation}. 
% However, the introduction of the newer ChIP protocols promotes simplification of the procedure. 
% Those simplifications enable the technique to be applied for a small amount of the cells and discover the genome-wide profiling of the target elements without prior knowledge of exact binding loci~\cite{collas2010current}. 

% DNA microarrays in combination with ChIP assay was the first approach to identify and analyze \textit{in-vivo} the protein-chromatin interactions of interest in broad genome coverage. 
% Such an approach is known as chromatin immunoprecipitation followed by genomic tiling microarray hybridization or simply ChIP-on-chip~\cite{ren2000genome, loden2005whole}. 
% This technology produces highly reproducible profiles.
% However, the large number of false positives and several limitations, such as genomic coverage, which is dependent on the microarray probe design, stimulated the development of other strategies. 

% Chromatin immunoprecipitation followed by sequencing, known as ChIP-sequencing or ChIP-seq, is a modern and relatively cheap method to analyze any protein associated with DNA. 
% Advances in high-throughput parallel sequencing technology and computational methods enable to generate and analyze extremely large data sets. 
% Unlike Chip-on-chip, ChIP-seq generally builds a profile with better resolution and signal-to-noise ratio, and detect more signals~\cite{park2009chip}.
% Also, the advantage of the ChIP-seq protocol is the small amount of the initial material~\cite{adli2010genome}.






%%%%%%%%%%%%%%%%%%%%%%%%%%%%%%%%%%%%%%%%%%%%%%%%%%%%%%%%%%%%
\section{Considerations of Experimental Design.}

\begin{figure}[b!]
    \centering
    \includegraphics[width=\textwidth]{../img/chip.jpeg}
    \captionsource{Overview of ChIP-seq experiment.}{www.abcam.com}
    \label{fig:graph_classes}
\end{figure}

A typical ChIP-seq protocol has many steps and requires a sufficient quantity of immunoenriched DNA. 
The consideration of the first step depends on the properties of the protein under investigation. 
For example, histone-DNA interactions are strong enough. 
Thus the fixation using formaldehyde as a crosslinking agent may not be necessary~\cite{barski2008identification}. 
On the other hand, proteins associated with DNA for a short time require a crosslinking step. 
In the case of histone deacetylases (HDACs) and histone acetyltransferases (HATs), an additional disuccinimidyl glutarate (DSG) treatment step before formaldehyde crosslinking demands to prevent protein-protein accosiation~\cite{wang2009genome}. 


Cross-linked chromatin is fragmented before ChIP. 
Fragmentation way is depending on the purpose of the experiment, cell type, number of cells, fixation conditions. 
For nucleosome modifications, MNase digestion may be preferred~\cite{kidder2011chip}.  
The method allows generating high-resolution data of mononucleosome sized particles. 
However, the nucleosome instability may cause the loss of signal.
To identify TF binding events, sonication of crosslinked chromatin is a preferable method. 
In this case, the micrococcal nuclease may cause degradation of the linker DNA~\cite{kidder2011chip}.
The sonication conditions should be optimized for each experiment type. 
A sonication buffer can influence the result~\cite{steger2008dot1l}. 
It is also important to avoid oversonication during the library preparation for transcription factors. 
Oversonication may lead to the deletion of protein epitopes from cross-linked chromatin~\cite{ostrow2015chip}. 
%\todo{Vysvětli, PROČ je to důležité. Navíc bych tu čekal nějakou citaci}



The successful ChIP experiment depends on the isolation of the protein under study from a complex mixture of the chromatin fragments and associated proteins. 
The target protein bound to the chromatin can be isolated from a sample using an antibody pre-immobilized onto insoluble magnetic beads that specifically recognizes that protein and that purification technique is known as immunoprecipitation (IP). 
After incubation, the immune complexes are separated from the lysate of non-interacting DNA by applying a magnetic field~\cite{slovakova2005use}.

The right choice of antibody is an essential factor for the ChIP-seq experiment. 
Many experiments are orienting on newly discovered proteins, which means that such a protein probably does not have any available antibody. 
However, the development and validation of the antibodies to each protein is pretty time-consuming and expensive~\cite{jarvik1998epitope}. 
For example, the number of proteins interacting with chromatin in a human cell is very high~\cite{ramani2005consolidating}, and the traditional approach may not be suitable.
Genetic engineering makes it possible to solve that problem by fusion of the short epitope sequence into a gene of interest for which an antibody is available~\cite{jarvik1998epitope,brizzard2008epitope,goldberg2010distinct}. 
This recombinant DNA method called epitope tagging and was first described in 1984~\cite{munro1984use}. 

%% \todo{to je příliš stručné - krátce vysvětli, co to je tagování}












%%%%%%%%%%%%%%%%%%%%%%%%%%%%%%%%%%%%%%%%%%%%%%%%%%%%%%%%%%%%55
\section{Library Construction and Sequencing.}

After chromatin purification with and without immunoprecipitation to prepare ChIP and corresponding input DNA fragments, the material is ready to be sequenced. 
The size range of $150$ to $300$ bp fragment length selection is equivalent to mono- and dinucleosome chromatin fragments~\cite{kidder2011chip}.

With the invention of the thermal cycler,  Sanger chain-termination sequencing method, which is referred to as first-generation sequencing (FGS), enables automatization of the sequencing process~\cite{quail2012tale}. 
The appearance of the pyrosequencing technique was the start of the next-generation sequencers (NGS). 
Unlike Sanger's sequencing, this method enabled real-time observation; and usage of nucleotides, which are not heavily modified~\cite{ronaghi1996real, ronaghi1998sequencing}.
Instead, the method uses luminescent labeling for pyrophosphate synthesis measuring. 
First, ATP sulfurylase converts pyrophosphate into ATP. 
The ATP is used as a substrate for the luciferase enzyme, which the light-emitting reaction. 
The light produced by the enzyme can be measured and is proportional to the amount of pyrophosphate. 
The measuring of the pyrophosphate produced during the reaction is used to identify the sequence~\cite{hyman1988new}. 
Like Sanger's, such a method is called sequencing by synthesis (SBS).

\paragraph{Sequencing by synthesis.}
The most common SBS method is Solexa/Illumina\textsuperscript{\texttrademark} platform~\cite{voelkerding2009next}. 
The bridge amplification produces clusters of clonal DNA. 
Fluorescent reversible-terminator (RT) nucleotides cannot bind further nucleotides due to protection at $3^\prime$ hydroxyl position~\cite{heather2016sequence}. 
During sequencing, free RT nucleotides ina competitive manner are able to complementary incorporate to the DNA template with the removal of the fluorescent tag, which is different for four bases and can be imaged by an analyzer~\cite{berglund2011next}. 
Illumina\textsuperscript{\texttrademark} HiSeq and MiSeq machine enable us to reach greater read length and depth. 
However, the recent NextSeq 500 technology reduces capturing time and cost by replacing the four-channel sequencing system with a two-channel system~\cite{reuter2015high}. 



The goal of the ChIP-sequencing is to obtain reads long enough to map uniquely to the reference. 
Multiple barcoded ChIP-seq libraries pooled together and sequenced in a single lane~\cite{craig2008identification} to reduce the cost of the experiment and produce high-quality data.
In many cases, 36-50 bp will be enough even for a complex organism such as a human. 
The longer reads sequenced, the deeper coverage per base may be achieved.
The main advantage of the Illumina\textsuperscript{\texttrademark} platform is the ability to obtain paired-end sequenced data due to clonal amplification. 
%Libraries can be sequenced using a single-end or paired-end strategy. 
Paired-end sequencing generates high sequencing coverage, improves alignment efficiency into repetitive regions, detects fragment size, and increases the probability of the alignment to a reference~\cite{kidder2011chip, chen2012systematic}.



\paragraph{Semiconductor sequencing.}
Another remarkable sequencing platform called IonTorrent\textsuperscript{\texttrademark} is also available on the market. 
Instead of using laser scanners, IonTorrent\textsuperscript{\texttrademark} technology measures pH during the sequencing without using labeled nucleotides. 
However, the homopolymeric regions may cause errors due to electronic signals corresponding to the number of released hydrogen ions~\cite{ambardar2016high}.
To improve the sequencing indel error rates and reduce GC-bias the technology comes up with a sequencing enzyme called Hi-Q~\cite{veras2014efficiency}. 










\section{ChIP-seq Read Mapping to the Reference}

\paragraph{Raw data.}
Raw data of short sequenced tags often appear in FASTQ format containing sequence information and quality scores, and may contain 10 to 60 million reads. 
And 60 million reads do not limit such files. 
In FASTQ files each entry is associated with 4 lines:


\begin{verbatim}
@SEQ_ID
GATTTGGGGTTCAAAGCAGTATCGATCAAATAGTAAATCCATTTGTTCAACTCACAGTTT
+
!''*((((***+))%%%++)(%%%%).1***-+*''))**55CCF>>>>>>CCCCCCC65
\end{verbatim}

First line contains sequence identifier and additional information.
Second line contains a short read in standard nucleotide code.
Third line always begins with '+' character.
And the last line encodes quality value for the short read from the second line.


\paragraph{Alignment.}
The sequence information from FASTQ files should be reconstructed by the alignment to the reference genome and find overlaps.
Several alignment tools have been developed based on extended Burrows-Wheeler transform (BWT). 
The algorithm was developed in 1994 as a data compression technique~\cite{li2009fast, siren2014indexing}.
And at the end of the 2000s was applied in the NGS field~\cite{simpson2010efficient}.
Suitable tools~\cite{langmead2009ultrafast, li2009fast, kim2019graph} map sequenced reads producing  SAM (Sequence Alignment Map), BAM (Binary Alignment Map), or relatively new CRAM output. 
BAM format is a widely used standard so far.

Many mapping software tools can search for exon junctions. 
However, unlike RNA-seq, the ChIP-seq experiment does not require to find spliced alignment.
Thus the optimal set of the parameters is to disable the spliced alignment and minimize the number of the allowed mismatches, increasing the number of the unique mapped reads and simplify further analysis~\cite{derrien2012fast}.

\paragraph{Mismatches.}
Alignment mismatches can be observed due to the biological differences between the genotype of the cultivated organism (indels and SNPs) and the reference genome~\cite{park2009chip}.
It is also possible to observe differences between reference and mapped tags due to contamination by the adapters or primers, which can be computationally filtered when a sequence is known~\cite{nakamura2011sequence}. 
Sequencing platforms may also produce sequence-specific errors (SSE). 
Some of them may be avoided or minimized by the improvement of the experimental procedure. 
The latest experiments tend to prolongate the number of cycles up to 150 bp. 
During the sequencing of the longer reads on the Illumina\textsuperscript{\texttrademark} platform may produce an error~\cite{nakamura2011sequence}.

\paragraph{Mapping sensitivity.}
Some sequences are able to fit more than the one location on the reference. Thus, allowing the random site for multiple mapped reads may increase the sensitivity of peak detection. 
The ration of the unique mapped reads over the total number of mapped reads is one of the parameters of the library quality assessment and should not be below 50\%~\cite{shin2013computational}.













%%%%%%%%%%%%%%%%%%%%%%%%%%%%%%%%%%%%%%%%%%%%%%%%%%%%%%%%%%%%%%%
\section{Quality control and computational analysis workflow.}

% The directionality of the sequencing reads produces bimodal enrichment on both strands centered around the binding site of the protein of interest. The cross-correlation metric can evaluate each obtained peak. The calculation is based on Pearson's linear correlation between forward and reverse strand for each complementarity base by shifting minus strand. The procedure generates two peaks, a bigger on corresponding to the fragment length, and a smaller one is associated with a read length. 

ChIP-seq data analysis is a powerful tool to get new insight into transcription control machinery and other biologic processes. 
%\todo{tohle je nicneříkající. Buď to vyhoď, anebo uveď konkrétní hodnoty: např. počet datasetů se každý rok zdvojnásobí..}
However, computational pipelines have not been straightforward; 
and new tools and analysis pipelines are needed to be developed.

NCBI's SRA and GEO~\cite{barrett2012ncbi} are the largest central public domains containing a vast amount of published genomic data, including ChIP-seq datasets. 
Many algorithms and tools have been introduced for addressing specific aspects of computational analysis. 
As was mentioned above, raw data often appear in fastq format. 
Processing pipelines are developed to make raw sequence reads annotated.
Low-quality data can be filtered before alignment to the reference. 
But such filtering is not necessary due to the inability of such sequences to be aligned~\cite{furey2012chip}.


\paragraph{Quality problems.}
Data quality problem is one of the most critical ChIP-seq experiment issues due to artifacts and noise reproducibility.
In a typical TF study reads mapped to the same genomic coordinate are filtered as redundant, because the expected number of mapped reads per genomic position is less than or equal to one. 
%% \todo{až teď mi došlo, že tenhle kousek vlastně nechápu "the expected number of mapped reads per base pair is less than 1."}
On the other hand, highly repetitive regions such as rDNA in yeast, long repetitive elements such as segmental duplication regions are the regions linked to important biological functions~\cite{nakato2017recent}. But the annotation of such regions is challenging and expect special algorithms that take such regions into account~\cite{chung2011discovering}.

\paragraph{Quality metrics.}
Data processing routine requires quality control on early step reducing further downstream analysis problems~\cite{ewels2016multiqc}.
To obtain reliable results essential to have a complex ChIP-seq library.
The complexity is measured by the non-redundant fraction (NRF), which is one of the QC-metrics.
The fraction is defined as the ratio between uniquely mapped reads over the total number of reads~\cite{landt2012chip}.

Another quality metrics to assess a ChIP sample is normalized cross-correlation (NSC) and relative cross-correlation (RSC) metrics of the fragment length and read length~\cite{landt2012chip, marinov2014large}. 
The directionality of the sequencing reads produces bimodal enrichment on both strands centered around the binding site of the protein of interest. 
The cross-correlation metric can evaluate each obtained peak. 
The calculation is based on Pearson's linear correlation between forward and reverse strand for each complementarity base by shifting minus strand. 
The procedure generates two peaks, a bigger on corresponding to the fragment length, and a smaller one is associated with a read length. 



%%%%%%%%%%%%%%%%%%%%%%%%%%%%%%%%%%%%%%%%%%%%%%%%%%%%%%%%%%%
\section{The best strategy}
\label{strategy}

Extracting useful information from huge data repositories combines techniques from computer science and statistics~\cite{friedman2001elements}. 
%\todo{místo "lore" bych použil "information" nebo "knowledge".}
In terms of ChIP-seq, the main goal is to distinguish significant enrichment events from background noise and to compare multiple profiles that are linked to the biological functions. 
The process of discovering patterns in large datasets is critically dependent on high data quality. 
Cell culture conditions, incubation, ChIP, and library construction may cause variability between datasets. 
To identify confident ChIP signals the presence of at least two biological replicates is necessary~\cite{kidder2011chip}. 
The availability of two or more replicates helps to assess whether the signal is a true biological event or just a technical variation. 

The ChIP-seq experiment should be consistent. 
Expected peak regions should have a similar signal profile, and that consistency is measured by the percentage of the overlapping peaks, correlation coefficient over selected intervals, and Irreproducible Discovery rate (IDR)~\cite{shin2013computational}.

For proper binding site detection, the fragmented genome is divided into two portions. 
The one portion undergoes al the immunoprecipitation procedure described above, whereas the other is sequenced directly. 
This direct sequencing is referred to as an input control dataset and used to normalize IP sequencing results~\cite{kidder2011chip}. 
Another control dataset can be generated using IP protocol, where antibodies cannot recognize a protein tagged with the epitope~\cite{flensburg2014comparison}. 
Both controls are informative and have advantages as well as disadvantages.
The choice of the right control dataset will be described more in detail in section~\ref{control}.
%\todo{popiš konkrétní vyhody a nevýhody těchto dvou typů kontrol, anebo odkaž čtenáře na kapitolu, kde to popisuješ}

In addition to both input and mock normalization methods, there is also spike-in normalization, in which foreign chromatin is used as internal control and helps avoid several biases~\cite{bonhoure2014quantifying}.
Spike adjustment allows comparing the occupancy level. 
By adding a constant low amount of foreign chromatin before immunoprecipitation, the method adjusts global changes. 
\chapter{Identification of enriched regions using peak calling}

\section{Building a signal profile.}

The computational method of the protein-chromatin binding event identification plays a central role in ChIP-seq analysis. 
After read mapping to the reference, the next step is to identify loci with high read density comparatively to the background, referred to ChIP-seq signals, or simply peaks.
More than 30 algorithms and tools have been implemented to solve that computational problem~\cite{chen2012systematic}.
The choice of the right peak caller is crucial and depends on the type of experiment~\cite{nakato2017recent}, e.g. TF binding event identification vs. long-range histone marks interactions.

Suitable software builds a peak profile along each chromosome. 
All the peak profiles can be divided into three categories~\cite{park2009chip}. 
Sharp peaks are typical for TF due to dependency on motif sequence. 
Histones have non-specific positioning on the DNA. 
Thus the peak profile is broad and can reach several kilobases. 
The third peak type is a mix of sharp and broad signals, a typical pattern for RNA polymerase II and transcription elongation factor~\cite{lin2011dynamic}.
% \todo{to je zavádějící. Nukleosomy sice nevážou konkrétní motiv DNA, ale vykazují jistou míru sekvenčních preferencí, která se liší v závislosti na organismu.}

The very first extended set methodology to calculate peak profile density was presented in 2007~\cite{robertson2007genome}. 
Fragments are sequenced from 5' to 3' end, and the minimal enough length of a sequenced read is 36 bp long. 
But the real fragment of DNA is longer, and thus the interaction of the protein of interest is somewhere on that size selected long DNA fragment. 
Each read is computationally extended in the 3' direction. 
Regions are scored by the number of overlap reads and assessed as a candidate peak.

Sequence directionality sets the stage for a smoothed profile. 
The strand-specific read distribution form bimodal pattern combined by shifting or extending tags toward the center~\cite{valouev2008genome}.

%%%%%%%%%%%%%%%%%%%%%%%%%%%%%%%%%%%%%%%%%%%%%%%%%%%
\section{Statistical model utilization for the assesment of the significance of estimated signals.}


Standard biological research has some rate of false positives. 
Due to there is no absolute proof or absolute rejection of the results in ChIP-seq studies, the analysis works with probabilities.
The end goal of the ChIP-seq experiment is the genomic loci of possible protein binding events. 
The end goal of the ChIP-seq experiment is to define the genomic loci of possible protein binding events. 
The candidate signal of a binding event is represented as a hypothesis $H_{1}$, and the null hypothesis $H_{0}$ is that there is no actual binding. To describe the statistical significance of the individual hypothesis, a P-value is calculated. 

The early approach was that the background noise is uniform; 
however, the usage of the control dataset shows that the different biases make uniform model is too ideal to be true~\cite{robertson2007genome}. 
That is why all peak calling algorithms make all the output signals associated with P-value~\cite{chitpin2019recap} to determine statistical significance in a hypothesis test. 
The null hypothesis's incorrect rejection produces the Type I error known as a "false positive". 
In the contest of ChIP-seq analysis, this type of error occurs when there is no actual binding event, but the peak caller shows that there is. 
The inversion of Type I error is Type II error, which is referred to as a "false negative". Considering the ChIP-seq experiment, the "false negative" error type is seen less serious than "false positives". 

To test peaks for significance, different peak calling algorithms adopt different statistical techniques. 
The widely used Poisson model was utilized in early software tools such as SICER~\cite{zang2009clustering}. 
The Poisson is directly connected to Binomial distribution:

\begin{align*}
    p_{k,n} = \binom{n}{k}p^k(1-p)^{n-k}
\end{align*}

where probability \textit{p} of succes in each trail and p$_{k,n}$  for \textit{k} succes in \textit{n} trails; but assosiated with rare events:

\begin{align*}
    p_{0,n} = (1 - p)^{n} = \left(1-{\frac{\lambda}{n}}\right)^{n} \to e^{-\lambda}
\end{align*}

\begin{align*}
    p_{1,n} = np(1 - p)^{n-1} = \frac{\lambda}{1-p}\left(1-{\frac{\lambda}{n}}\right)^{n} \to \lambda e^{-\lambda}
\end{align*}

\begin{align*}
    p_{2,n} = \frac{1}{2} n(n - 1) p^{2} (1-p)^{n-2} = \frac{1}{2} \frac{\lambda^{2} - \lambda p}{ (1-p)^{2}} \left(1-{\frac{\lambda}{n}}\right)^{n} \to \frac{1}{2} \lambda^{2} e^{-\lambda}
\end{align*}

For \textit{k} succes in \textit{n} trails with probability p=${\lambda / n}$ , the binomial probability p$_{n,k}$ approaches the Poisson probability:

\begin{align*}
    P_k = \frac{\lambda _i ^{k}}{k!} e^{- \lambda _i}
\end{align*}

The Poisson parameter $\lambda_i$ is supposed to be constant across the genome and provided to be inadequate for ChIP-seq peak calling. 
And the negative binomial model was suggested by CisGenome~\cite{ji2008inte}.

\begin{align*}
    NB_{y_i, \mu _i, \alpha} = \frac{\Gamma (y_i + \alpha ^{-1})}{\Gamma(\alpha ^{-1})\Gamma(y_i + 1)} \left(\frac{1}{1 + \alpha \mu_i}\right) ^{\alpha ^{-1}} \left(\frac{\alpha \mu _i}{1 + \alpha \mu _i}\right) ^{y _i}
\end{align*}

where 

\begin{align*}
    \mu _i = t _i \mu
\end{align*}

\begin{align*}
    \alpha = \frac{1}{\nu}
\end{align*}

Another suggestion was to estimate $\lambda$ for each genomic position by the local Poisson model. 
Such an approach was introduced by the most popular peak caller called MACS~\cite{zhang2008model}.
The tool slides with a constant size window across the genome, merge overlapping peaks by extending the read. 
The highest tag pileup is defined as a summit of a signal.  

\section{Multiple hypothesis testing.}

After scanning through the genome and find a large number of candidate regions, the quality of the detected peaks should be detect.
Suppose we have \textit{n} genomic loci obtained.
The $i^{th}$ null hypothesis $H_{0,i}$ with corresponding P-values $p_i$.
The global null hypothesis of simultenious test of all null hypitheses  is defined:

\begin{align*}
    H_0 = \displaystyle\bigcap_{i=1}^{n} H_{0, i}
\end{align*}

Fisher's combined probability test combines known P-values using:

\begin{align*}
    T = - \displaystyle \sum_{i=1}^{n} 2 \log p_i \sim  \chi_{2n}^{2}
\end{align*}

Bonferroni's test looks at the smallest P-value and at given desired level $\alpha$ tests the global null hypothesis by testing each $H_{0,i}$ at level $\alpha /n$. 
And whenever $p_i \leq \alpha / n$ rejects the global null hypothesis.
Assuming the global hypothesis is true, the overal level control is: 

\begin{align*}
    P_{H_0}(Error Type I) = P_{H_0} \left[\bigcup_{i=1}^{n} \left\{ p_i \leq \alpha / n\right\}\right] \leq \sum_{i=1}^{n} P_{H_0}(p_i \le \alpha / n)  = n \frac{\alpha}{n} = \alpha
\end{align*}

Even though both tests are easy and straightforward, non of them is effective. 
Fisher's test will be eliminated in a large number of the true null hypothesis. 
In the case of  Bonferroni's test, the only one P-value is used. 
Hence it can be applied only if very few binding events are expected to be significant. 
The rejection of extreme value of binding event at the $\alpha$ significance level leads to increase number of false positives. 
The idea of the global null hypothesis rejection is not suitable for genomic analysis. 
Another approach is the familywise error (FWE) rate method, which controls the probability of performing one or more Type I errors. 
However, the method is very conservative and does only a few rejections. 

The application of the False Discovery Rate (FDR) approach in the genomic field is associated with microarray technology~\cite{lai2017statistical}.
The method allows a few small rejections if the majority of the rejections are correct. Such testing adjusts the statistical confidence based on the number of tests. 

%%%%%%%%%%%%%%%%%%%%%%%%%%%%%%%%%%%%%%%%%%%%%
\section{Statistical sin}
\chapter{bias, artefacts}

% To ensure reliability of the data, at least duplicate biological replicate experiments should be done.
% Sonication of crosslinked chromatin may be preffered method.
% Mocrococcal nuclease degrade linker DNA where TF tend to bind.
% The conditions of sonication should be optimized for each cell type.
% Because they depend on cell type, type of sonicator and  sonicator settings.

\section{assay artefacts}
Strong enrichment signals suggestive of proteins binding to genomic loci where genes were higly transcribed were found.
Moreover, the enrichment for proteins binding to higly-transcribed genes was was observed even in controls like moch ChIP-seq data.
Which poins to an overall bias that could contaminate any ChIP-seq data with false positives~\cite{park2013widespread}
A secondary bias of nucleosomal periodicity was also commonly observed across ChIP-seq dataset.
And contributed additional false positives in which proteins falsly appeared to interact with nucleosomes.

Two features among strong false positive signals.
First, the signals were present within gene bodies.
Second, Stronges signal derived from genes thet are known to be higly expressed.

% It is possible that certain TF trully bind to ORFas a means of regulating gene expression.
A common use of ChIP-seq is to examine binding of a given factor under different growth conditions or backgrounds.
Since only a single variable is changed (growth condition), it might be assumed that comparing binding under different conditions offers a reliable means of identifying biologically relevant targets, with most background artefacts being ormalized out.

One bias arised during genome sonication.
Open chromatin regions are easely shared than other regions.
Thus these open regions yield more protein-DNA complexes.
IP step immunoprecepitate more complexes from the open chromatin regions.
And as a result gives more sequenced reads.

To correct this bias, the fragmented genome are divided into two portions.
One portion goes through the IP step.
And then sequenced.
Other portion is sequenced directly.
This portion serve the input control.

This input control can be used to normalize the shearing bias of sonication~\cite{kharchenko2008design}.
% Type of normalization controls might be appropriate for normalizing false positives?

In the analysis of ChIP-seq data two types of normalization or correction controls are commonly used.

The input sample has the advanage that all the regions of the genome are well represented.
The sample concentration is ample and stable for constructing sequencing libraries.
The same sample can potentially serve as the control for several related experiments.

The input a baseline signal for reads across the genome, factoring in sequence mappability and copy number differences relative to the reference genome.
For these reasons, input has been suggested as a more effective control.

However, genes transcribed at high rates is not adequately represented in the input.

A mock ChIP is designed to correct data with large quantity of spurious sites in ChIP-seq, which are coused by uneven genomic sonication and nonspecific interactions between chromatin and antibody.
Whereas DNA input controls corrects only for uneven sonication.

Mock better reflects the background enrichment from highly transcribed genes.
Mock ChIP-seq data exhibit a stronger expression bias than the corresponding input sample.
Therefore correction by mock ChIP (normalization) would be more effectively reduce the false-positives than normalization by input~\cite{park2013widespread}.
Mock ia s better control for minimizing the appearence of occupancy signal over transcribed regions.

Measuring the binding of a TF under two different conditions and identifying the differentially bound target offers the most reliable way to identify targets of biological significence.
The assumption that most sources of background signal are canseled out between two samples is risky.
Expression biase derives directly from actively transcribed genes.
Transcription will differ between two conditions.
Even in these cases ChIP data from each condition has to be properly corrected by the corresponding mock ChIP data to minimize false positives.

\section{crosslink bias}

Source and mechanism of the background expression bias arise from direct or indirect non-specific interactons of the immunoprecepitated protein with DNA in  open chromatin at highly transcribed regions, trapped by the crosslinked process.
It is unclear why the phenomenon exists in mock ChIP datasets.
It is possible that even low level non-spesific interactions between the antibody and cross-reacting cellular proteins contrebute to this phenomenon.
Or Open chromatin shows oreferential recovery through the immunoprecepitation process.
Highly expressed genes are characterized by open chromatin conformation, nucleosome perturbation and DNA more exposed to interactions.\cite{}
Although the antibody in IP binds specifically to its target TFs.
It can also bind non-specifically to other proteins.
Under some cross-linkin conditions, this regions might cause the hyper ChIPability\cite{}

One possible solution is to develope alternative approaches that avoid the cross-linking step.
The lack of cross-linking necessarily means that only proteins very tightly assosiated with the chromatin can be immunoprecepitated.
Non-specific cross-linking in human cell line is not an issue if formaldehyde treatment is limited to 10 min or less~\cite{}.
Shortering the fixation time reduces the background level due to non-specific protein binding to trap preferentially native DNA-proteins interactions~\cite{baranello2016chip}.
Under some conditions cross-linked artefacts might be relevant.
Good biochemical practice and proper controls can detect and minimize this bias.

To control for nonspecific antibody bias, a mock ChIP can be utilized.

However, mock ChIP yields much less DNA material than DNA input.
And mock is more susceptible to technical noise~\cite{kidder2011chip, landt2012chip}.

%%%
\section{GC content bias in ChIP-seq is challenging}

Fragmentation is the first step of DNA sequencing process.
Using enzymatic digestion, chemical shearing or different mechanical method, purified DNA is broken up into short fragments.
The sequenced reads must overlap to achieve the required level of resolution for a ChIP experiment the genome regions should be sequenced multiple times.
Whereby the experiment requires many copies of a whole genome.
Nowadays ChIP purified DNA is obtained from many cells.
Multiple genome copies provide a sufficient number of overlapping reads after cleavage and sequencing.
In addition, sigle-cell sequencing requires PCR-amplification before the fragmentation step~\cite{clark2016single}.

The early assumption was that genomic DNA break randomly.
However, the experiment~\cite{poptsova2014non} with different methods of DNA fragmentation showed that the rates of double-stranded breaks are sequence-dependent.
Which can lead the source of the GC-bias have been observed in several NGS experiments~\cite{benjamini2012summarizing}~\cite{dohm2008substantial}. 

The coverage for ChIP-seq varies across experiments due to GC-content bias.
And that variability leads obtain false-positive signals.
However, true binding sites of the protein of interest are expected to occour in high GC regions.
Those regions have biological relevance such as promotore regions.
Modelling of the GC-content at the fragment level~\cite{benjamini2012summarizing} is the optimal approach, but is not directly applicable to ChIP-seq analysis.

The mixed-model method for the GC-content bias correction was presented~\cite{teng2017accounting}.
Incorporation of the approach into current peak callers shows substantial improvements in signal finding.
However, the method is not suitable for broad peak identification as histone modifications.

Also new methods of cleavege of the purified DNA cleavage for library construction are developing~\cite{}, which provide better uniform coverage for GC reagions.


% It is troubling that the number of reads mapped to a genome region depends on the sequence itself.

% Windows with a GC content of 40\% contain almost twice as many reads as windows with lower content~\cite{dohm2008substantial}.
% This variability does not reflect the signal of interest, but might confuced it.
% Since GC abundance is heterogeneous across the genome and often correlated with functionality, the GC effect can be hard to tell apart from the true signal.
% The effect is not consistent between repeated experiments, or even libraries within the same experiment.
% GC counts could be associated with the stability of the DNA, and thus modify the probability of a fragmentation point occurring in the genome.
% Binding sites are expected to occur in or near high GC-content regions such as gene promoters.
% % IDR algorithms to select peaks
% GC-content bias lead peak callers to report a substantial number of false positives.

% Most peak callers operate on bin level information.
% Algorithms define bins, calculate coverage measurements.
% % something about new approach using a mixed model, which permits independent adjustments of the signal and back- ground signals and thus circumvents the confounding challenge and can be incorporated into most current peak callers



%\chapter{Practical part}

\section{Objectives}

\section{Formats and Methods}

\section{Results}

\chapter*{Conclusion}
\addcontentsline{toc}{chapter}{Conclusion}

In the first chapter of the thesis we made a brief overview of ChIP-seq experimental design, and its computational analysis.
This method had been using for a decade, however, it is still a powerful tool for understanding processes of the genome of interest.


The second chapter was about a computational tool called peak calling.
In this section was mentioned basic statistics utilized in peak calling evaluation. 
Also, we mentioned popular peak calling tools based on a different statistical model.
It is also good to know that even a good designed tool has its limitations and may give rise to false-positive signals.

The last section was about known causes of false signals, which may be obtained during a ChIP-seq experiment.
Some false signals are caused by the nature of the genome. 
Others are raised due to sequential and computational issues.
This section also contains a theoretical description of possible solutions. 


%%% Bibliography
\include{bibliography}

%%% Figures used in the thesis (consider if this is needed)
\listoffigures

%%% Tables used in the thesis (consider if this is needed)
%%% In mathematical theses, it could be better to move the list of tables to the beginning of the thesis.
% \listoftables



%%% Attachments to the bachelor thesis, if any. Each attachment must be
%%% referred to at least once from the text of the thesis. Attachments
%%% are numbered.
%%%
%%% The printed version should preferably contain attachments, which can be
%%% read (additional tables and charts, supplementary text, examples of
%%% program output, etc.). The electronic version is more suited for attachments
%%% which will likely be used in an electronic form rather than read (program
%%% source code, data files, interactive charts, etc.). Electronic attachments
%%% should be uploaded to SIS and optionally also included in the thesis on a~CD/DVD.
%%% Allowed file formats are specified in a provision of the rector no. 72/2017.
\appendix
%\chapter{Attachments}

%\section{First Attachment}
%s
\openright
\end{document}
